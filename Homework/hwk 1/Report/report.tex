\documentclass[conference]{IEEEtran}
\IEEEoverridecommandlockouts
% The preceding line is only needed to identify funding in the first footnote. If that is unneeded, please comment it out.
\usepackage{cite}
\usepackage{amsmath,amssymb,amsfonts}
\usepackage{algorithmic}
\usepackage{graphicx}
\usepackage{textcomp}
\usepackage{xcolor}
\usepackage{float}
\def\BibTeX{{\rm B\kern-.05em{\sc i\kern-.025em b}\kern-.08em
    T\kern-.1667em\lower.7ex\hbox{E}\kern-.125emX}}
\begin{document}

\title{HOMEWOEK 1}

\author{\IEEEauthorblockN{Runlin Hou}
\IEEEauthorblockA{\textit{ECE, School Of Graduate Studies} \\
\textit{Rutgers University}\\
hourunlinxa@gmail.com}
}

\maketitle

\section*{problem 1}
According to the data, the three points with the smallest L2 distance to test data is.
\[\begin{aligned}
    L2_{A1}&=\sqrt{(0-1)^2+(1-0)^2+(1-1)^2}=\sqrt{2}\\
    L2_{C2}&=\sqrt{(0-1)^2+(-1-0)^2+(1-1)^2}=\sqrt{2}\\
    L2_{A0}&=\sqrt{(0-1)^2+(1-0)^2+(0-1)^2}=\sqrt{3}
\end{aligned}\]
When $K=1$, the point chose to decide the class of the test data might be A1 or C2. So 
test data might be classified to be A or C in a same probability. \\
When $K=2$, the points chose to dicide the class of the test data would be A1 and C2,
which means test data got the same probability to be classified in A and C.\\
When $K=3$, all the three points would be chose to decide the class of test data. So the
test data would be classified to be A.

\section*{problem 2}


\end{document}